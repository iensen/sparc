\documentclass[a4paper,10pt]{article}
\usepackage[utf8]{inputenc}

%opening
\title{}
\author{}

\begin{document}
\textbf{INPUT:} Formula F of the form $G_1 \land \dots \land G_n$, where each $G_i$ can be of the form:
\begin{itemize}
 \item $t_1~\diamond~t_2$. Possible forms of $t_1, t_2$ and $ \diamond$ are presented in the table: \\
  \begin{minipage}{20cm}
  \begin{tabular}{ |c | c | c| }
  \hline                        
  $t_1$ & $t_2$ & $\diamond$ \\
  \hline
  arithmetic term\footnote{for definition of arithmetic term, see algorithms.pdf} & arithmetic term & $\{ <,<,\leq,\geq,=,\neq \}$ \\
  \textit{record} & \textit{record} & $\{ =,\neq \}$ \\
  \textit{record} & \textit{symbolic constant} & $\{\neq \}$ \\
  \textit{record} & \textit{arithmetic term} & $\{\neq \}$ \\
  \textit{variable} & \textit{record} & $\{ =,\neq \}$ \\
  \textit{variable} & \textit{symbolic constant} &$\{ <,<,\leq,\geq,=,\neq \}$ \\
  \textit{variable} & \textit{arithmetic term} &$\{ <,<,\leq,\geq,=,\neq \}$ \\
  \hline  
\end{tabular}
 \end{minipage}

 \item $t \in D$. Possible forms of $t$ and $D$ are presented in the table : \\
   \begin{minipage}{20cm}
  \begin{tabular}{ |c | c | }
  \hline                        
  $t$ & $D$  \\
  \hline
  \textit{arithmetic term} & $n_1..n_2$ , $\{n_1,\dots, n_k\}$, where $n_1,\dots, n_k$ are numbers \\
  \textit{non-arithmetic variable}\footnote{for definition of arithmetic variable, see algorithms.pdf} & $\{t_1,\dots,t_n\}$, where all $t_i$ are arbitrary ground terms.  \\
 % record & symbolic constant & $\{\neq \}$ \\
 % record & arithmetic term & $\{\neq \}$ \\
 % variable & record & $\{ =,\neq \}$ \\
 % variable & symbolic constant &$\{ <,<,\leq,\geq,=,\neq \}$ \\
 % variable & arithmetic term &$\{ <,<,\leq,\geq,=,\neq \}$ \\
 % \hline  
\end{tabular}
 \end{minipage}
 \item $\neg (t\in D)$
  the possible forms of t and D are the same as in $ (t\in D)$.
\end{itemize}

\textbf{OUTPUT:} True if $F$ is satisfiable and false otherwise.

\textbf{BEGIN}.
$BODY := {\bf true}$;
$\Pi_{prolog} := \emptyset$

\textbf{For each} $G_i$ in $G_1 \land \dots \land G_n$
\begin{enumerate}
 \item \textbf{$G_i$   of the form $t_1 \diamond t_2$ or $\neg(t_1 \diamond t_2)$} 

  $BODY := BODY \land G_i$ 


 \item\textbf{$G_i$   of the form $(t\in D)$}
\begin{enumerate} 
 \item \textbf{t is an arithmetic term, and D is of the form n1..n2}

   $BODY := BODY \land  t~in~n1..n2$ 

 \item \textbf{t is an arbitrary term, D is  of the form $\{t1,t2,\dots,tn\}$}

    $\Pi_{prolog} := \Pi_{prolog} \cup set\_d(X):-member(X,[t1,t2,\dots,tn]).$\footnote{if t is arithmetic, we preprocess D and remove all non-numbers to avoid errors}
  
    $BODY := BODY \land set\_d(t)$ 

\end{enumerate}

\item \textbf{$G_i$   of the form $\neg(t\in D)$,}



\begin{enumerate}
\item \textbf{t is an arithmetic term and $D$ is of the form $n1..n2$}
      
      Let $g_i$ is an unique label for $G_i$
      
      $\Pi_{prolog} := \Pi_{prolog} \cup g_i(t) :- n\#>n_2. \cup g_i(t) :- n\#<n_2$.
  
      $BODY := BODY \land  g_i(t)$ 

\item \textbf{t is an arbitrary term, D is  of the form $\{t1,t2,\dots,tn\}$}

 $\Pi_{prolog} := \Pi_{prolog} \cup set\_d(X):-member(X,[t1,t2,\dots,tn]).$\footnote{same as above}
  
    $BODY := BODY \land ~~\backslash+ set\_d(t)$ 
    
\end{enumerate}

\end{enumerate}
\textbf{End for} 


\noindent
Let $V_1,\dots V_n$ be all arithmetic variables in $G_1,\dots,G_n$

$BODY:-REORDER(BODY) \cap labeling([],[V_1,\dots V_n])$.

$\Pi_{prolog} := \Pi_{prolog}  \cap  p :-BODY.$

If the answer to query ?-p to program  $\Pi_{prolog}$ is 'yes', 

return true; else return false. 

\vspace{0.3cm}
\noindent
\textbf{END}
\end{document}



